\def\paperversiondraft{draft}
\def\paperversionblind{blind}

\ifx\paperversion\paperversionblind
\else
  \def\paperversion{blind}
\fi

% !TEX TS-program = pdflatex
% !TEX encoding = UTF-8 Unicode

% This is a simple template for a LaTeX document using the "article" class.
% See "book", "report", "letter" for other types of document.

\documentclass[12pt]{article} % use larger type; default would be 10pt

\usepackage{longtable}
\usepackage{booktabs}
\usepackage{xargs}
\usepackage{xparse}
\usepackage{xifthen, xstring}
\usepackage{ulem}
\usepackage{xspace}
\usepackage{multirow}
\setlength {\marginparwidth }{2cm}
\usepackage{todonotes}
\bibliographystyle{amsalpha}
\makeatletter
\font\uwavefont=lasyb10 scaled 652
\DeclareSymbolFontAlphabet{\mathrm}    {operators}
\DeclareSymbolFontAlphabet{\mathnormal}{letters}
\DeclareSymbolFontAlphabet{\mathcal}   {symbols}
\DeclareMathAlphabet      {\mathbf}{OT1}{cmr}{bx}{n}
\DeclareMathAlphabet      {\mathsf}{OT1}{cmss}{m}{n}
\DeclareMathAlphabet      {\mathit}{OT1}{cmr}{m}{it}
\DeclareMathAlphabet      {\mathtt}{OT1}{cmtt}{m}{n}
%% \newcommand\colorwave[1][blue]{\bgroup\markoverwith{\lower3\p@\hbox{\uwavefont\textcolor{#1}{\char58}}}\ULon}
% \makeatother

% \ifx\paperversion\paperversiondraft
% \newcommand\createtodoauthor[2]{%
% \def\tmpdefault{emptystring}
% \expandafter\newcommand\csname #1\endcsname[2][\tmpdefault]{\def\tmp{##1}\ifthenelse{\equal{\tmp}{\tmpdefault}}
%    {\todo[linecolor=#2!20,backgroundcolor=#2!25,bordercolor=#2,size=\tiny]{\textbf{#1:} ##2}}
%    {\ifthenelse{\equal{##2}{}}{\colorwave[#2]{##1}\xspace}{\todo[linecolor=#2!10,backgroundcolor=#2!25,bordercolor=#2]{\tiny \textbf{#1:} ##2}\colorwave[#2]{##1}}}}}
% \else
% \newcommand\createtodoauthor[2]{%
% \expandafter\newcommand\csname #1\endcsname[2][\@nil]{}}
% \fi


%%% Examples of Article customizations
% These packages are optional, depending whether you want the features they provide.
% See the LaTeX Companion or other references for full information.

%%% PAGE DIMENSIONS
\usepackage{geometry} % to change the page dimensions
\geometry{a4paper} % or letterpaper (US) or a5paper or....
\geometry{margin=1in} % for example, change the margins to 2 inches all round
% \geometry{landscape} % set up the page for landscape
%   read geometry.pdf for detailed page layout information

\usepackage{graphicx} % support the \includegraphics command and options

% \usepackage[parfill]{parskip} % Activate to begin paragraphs with an empty line rather than an indent

\usepackage[utf8x]{inputenc}
\usepackage{amssymb}
\usepackage{listings}

\usepackage{color}
\definecolor{keywordcolor}{rgb}{0.7, 0.1, 0.1}   % red
\definecolor{commentcolor}{rgb}{0.4, 0.4, 0.4}   % grey
\definecolor{symbolcolor}{rgb}{0.0, 0.1, 0.6}    % blue
\definecolor{sortcolor}{rgb}{0.1, 0.5, 0.1}      % green
\usepackage{listings}
\def\lstlanguagefiles{lstlean.tex}
\lstset{language=lean}


%%% PACKAGES
\usepackage{inputenc}
\usepackage{booktabs} % for much better looking tables
\usepackage{array} % for better arrays (eg matrices) in maths
\usepackage{paralist} % very flexible & customisable lists (eg. enumerate/itemize, etc.)
\usepackage{verbatim} % adds environment for commenting out blocks of text & for better verbatim
\usepackage{subfig} % make it possible to include more than one captioned figure/table in a single float

\usepackage{textcomp}


% These packages are all incorporated in the memoir class to one degree or another...

%%% HEADERS & FOOTERS
\usepackage{fancyhdr} % This should be set AFTER setting up the page geometry
\pagestyle{fancy}
\renewcommand{\headrulewidth}{0pt} % customise the layout...
\lhead{\leftmark}\chead{}\rhead{\rightmark}
\lfoot{}\cfoot{\thepage}\rfoot{}

%%% SECTION TITLE APPEARANCE
\usepackage{sectsty}
\allsectionsfont{\sffamily\mdseries\upshape} % (See the fntguide.pdf for font help)
% (This matches ConTeXt defaults)

%%% ToC (table of contents) APPEARANCE
\usepackage[nottoc,notlof,notlot]{tocbibind} % Put the bibliography in the ToC
\usepackage[titles,subfigure]{tocloft} % Alter the style of the Table of Contents
\renewcommand{\cftsecfont}{\rmfamily\mdseries\upshape}
\renewcommand{\cftsecpagefont}{\rmfamily\mdseries\upshape} % No bold!

\setlength{\parindent}{0em}
\setlength{\parskip}{1em}
\usepackage{amsmath}
\usepackage{amsthm}
\usepackage{upgreek}
\usepackage{tikz-cd}
\theoremstyle{definition}
\newtheorem{thm}{Theorem}[subsection]
\theoremstyle{definition}
\newtheorem{corol}[thm]{Corollary}
\theoremstyle{definition}
\newtheorem{lemma}[thm]{Lemma}
\theoremstyle{definition}
\newtheorem{defn}[thm]{Definition}
\newtheorem{exmpl}[thm]{Example}
\usepackage{lscape}
\usepackage{hyperref}
\usepackage{titlesec}
\newcommand{\invamalg}{\mathbin{\text{\rotatebox[origin=c]{180}{$\amalg$}}}}

\setcounter{secnumdepth}{4}

\titleformat{\paragraph}
{\normalfont\normalsize}{\theparagraph}{1em}{}
\titlespacing*{\paragraph}
{0pt}{3.25ex plus 1ex minus .2ex}{1.5ex plus .2ex}

%%% END Article customizations

%%% The "real" document content comes below...

\title{Using Universal Properties to Prove Equalities of Morphisms Automatically}
\author{Christopher Hughes}

\begin{document}

We have implemented a verified algorithm in Lean that will equality of
expressions on sequences of bits of arbitrary length. Semantic equality
of expressions in the following syntax is decidable when interpreted
as operations on infinite sequences of bits. This implies
decidability of equality on finite sequences of arbitrary length.

\begin{lstlisting}[language=lean]
  inductive term : Type
  | var : ℕ → term
  | zero : term
  | neg_one : term
  | one : term
  | and : term → term → term
  | or : term → term → term
  | xor : term → term → term
  | not : term → term
  | ls : term → term
  | add : term → term → term
  | sub : term → term → term
  | neg : term → term
  | incr : term → term
  | decr : term → term
\end{lstlisting}

We prove decidability by proving that an equality of terms $t_1$ and $t_2$ is true
for words of arbitrary length if it is true for words of length $2^n$,
where $n$ is the total number of operations requiring a carry in either of the terms.
The operations requiring a carry are \lstinline{one}, \lstinline{add}, \lstinline{sub},
\lstinline{ls}, \lstinline{incr}, \lstinline{decr} and \lstinline{neg}.
The equality can then be checked by a brute force check over all words of length $2^n$.
For all of the equalities in the list (is the list in the Zulip thread somewhere else in the paper?),
$n$ is no more than $3$ and so they can be verifying by checking on all $8$-bit words.

Use the notation $S$ to denote the set of all infinite sequences of bits,
i.e. the set of function $\mathbb{N} \rightarrow \{0,1\}$. For the set of
words of length $n$ we use the notation $W_n$

% We will first define a function $thingy$, which for any natural number $c$
% and $n$ takes as its input an initial carry $i \in W_c$ and two functions,
% $carry : W_{c} \times W_n \rightarrow W_{c}$ and $bit : W_{c} \times W_n \rightarrow W_1$,
% and returns a function $S^n \rightarrow S$, and a sequence .

% The function $thingy$

We will show that all functions $S^n \rightarrow S$ definable with \lstinline{term}
can be written as a finite state machine. Given a number of carry bits $c$ and
an arity $n$, an initial state $i \in W_c$ a transition function
$t : W_{c} \times W_n \rightarrow W_{c}$,
an output function $o : W_c \times W_n \rightarrow \{0, 1\}$, we can define
a function $thingy : S^n \rightarrow S$ as follows. If $w \in S^n$ then

\begin{equation}
  thingy(i, t, o, w, n) = o(carry(n), w_{n})
\end{equation}

where $carry(n)$ is defined inductively as follows

\begin{equation}
  \begin{aligned}
    carry(i, t, o, w, 0) = i \\
    carry(n+1) = t(carry(n), w_{n})
  \end{aligned}
\end{equation}

Later, we will prove that all terms in our syntax can be written as operations using $thingy$.
First, we will show that given an operation defined using thingy, we can check in finite
time if it returns a constant sequence of zeros on all inputs $w \in S^n$.

Lemma 1

For any word $w$ there and set of natural numbers $I$ of cardinality greater
than $2^c$, there exist $x, y \in I$ such that $m \ne n$ and
$carry(i, t, o, w, x) = carry(i, t, o, w, y)$. We prove this using a
cardinality based argument, if there were not such $m$ and $n$,
then the function sending $k \in I$ to $carry(i, t, o, w, k)$ would be injective.
But $I$ has cardinality at least $2^c + 1$,
and the set of words of length $c$ has cardinality $2^c$, so the function cannot
be injective. Contradiction.

Theorem 2

For any word $w$ and natural number $n$, such that
$thingy(i, t, o, w, n)$ there exists another word $w'$ and
a natural number $m < 2^c$ such that $thingy(i, t, o, w, m)$ is also equal to $1$.

Proof of Theorem 2

We prove this by strong induction on $n$.
If $n < 2^c$ this is straightforward. So assume $n \ge 2^c$. From
Theorem 1 we know that there are $x, y \le n$ such that $x \ne y$,
and $carry(i, t, o, w, x) = carry(i, t, o, w, y)$.
Without loss of generality, assume $x < y$. We can then define a word $w'$
by cases. If $k < x$ then $w'_k = w_k$, if $k \ge x$ then $w'_k = w_{k + y - x}$.
Note that for any $k \ge 0$ we have $w'_{x+k} = w{y+k}$.
We know that $carry(i, t, o, w', x) = carry(i, t, o, w, y)$, since
$w$ and $w'$ are equal on all inputs $k < x$, and the $carry$
at $x$ only depends on the input $w'$ at values less than $x$.

Now note that for any $k > 0$ we have $thingy(i, t, o, w', x+k) = thingy(i, t, o, w', y+k)$,
and $carry(i, t, o, w', x+k) = carry(i, t, o, w', y+k)$. This is straightforward to
prove by induction on $k$, using the fact that $w'_{x+k} = w_{y+k}$ and
$carry(i, t, o, w', x) = carry(i, t, o, w, y)$.

Therefore $thingy(i, t, o, w', x + n - y) = thingy(i, t, o, w, n)$, and since
$x + n - y < n$ theorem 2 follows from applying the induction hypothesis to
$x + n - y$.

From Theorem 2 it is clear that if $thingy(i, t, o, w, n)$ is zero for
any sequence $w$ at any position $n < 2^c$, then it is zero for all $w$ and $n$.

We now show that every term can be written as an operation using $thingy$.

All of the operations in \lstinline{term} can be defined suing $thingy$. For example,
to define addition, there is one carry bit, the inital state in $0$, the transition
function $t : W_3 \rightarrow \{0, 1\}$ sends $a$ to $(a_0 \& a_1) \| (a_1 \& a_2) \| (a_0 \& a_2)$
and the output function $o : W_3 \rightarrow \{0, 1\}$ sends $a$ to $a_0 \oplus a_1 \oplus a_2$.
To define the bitwise operations we use $thingy$ with zero bits of carry.
The left shifting operation \lstinline{ls} requires one bit of carry. To define
$one$ we use $thingy$ with one bit of carry and arity $0$,
the initial state is $1$, the transition function $t : \{0, 1\} \rightarrow \{0, 1\}$ sends
everything to $0$. The output function $o : \{0, 1\} \rightarrow \{0, 1\}$ sends $a$ to $a$.

We can also compose operations written using $thingy$ into an operation using $thingy$.
We will show this for arity two operations, but the method generalizes to higher arity.
Suppose we have an arity two operation $f : S^2 \rightarrow S$ defined using thingy with
$c$ carry bits, initial state $i$, transition function $t$ and output function $o$.
Suppose we also have two arity $n$ operations $g_1 : S^n \rightarrow S$ and $g_2 : S^n \rightarrow S$,
also defined using thingy with $c_0$ and $c_1$ carry bits,
initial states $i_0$ and $i_1$, transition functions $t_1$ and $t_2$ and output functions $o_1$ and $o_2$.
Then we want to find an expression of $W \mapsto f(g_1(w), g_2(w))$ in terms of $thingy$.

The number of carry bits $d$ of the new expression is $c + c_0 + c_1$.
We now view $W_{c + c_0 + c_1 + n}$ as a the product of sets $W_c \times W_{c_0} \times W_{c_1}$.
The new initial state $j$ is $(i, i_0, i_1)$.
The new transition $u$ sends $(a, a_0, a_2, w)$ to $(t(a, w), t_1(a_0, w), t_2(a_1, w))$.
The new output function $p$ sends $(a, a_0, a_1, w)$ to $o(a, o_0(a_0, w), o_1(a_1, w))$.

Using this result if $t_1$ and $t_2$ are terms that can be written as operations
using $thingy$ then so is $t_1 + t_2$ and $t_1 \& t_2$ and so on. Note that the
number of carry bits required to express $t$ in terms of $thingy$ is the
total number of operation in $t$ that require a carry. Given $t_1$ and $t_2$, then
$t_1$ is semantically equal to $t_2$ iff $t_1 \oplus t_2 = 0$.
We have shown $t_1 \oplus t_2$ can be expressed using $thingy$ and we have already
shown that it is posisble to check if a $thingy$ is zero on all inputs.
Therefore verifying $t_1$ is semantically equal to $t_2$ is decidable.


\end{document}